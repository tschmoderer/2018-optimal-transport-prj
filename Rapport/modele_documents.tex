% Pour le bon support de la langue française : 
\usepackage[utf8]{inputenc}
\usepackage[T1]{fontenc}
\usepackage[francais]{babel}

\usepackage{lipsum} % \lipsum créera un texte test
\usepackage[margin=2.4cm]{geometry} %pour des marges

\usepackage{framed} % des contours
\usepackage[framed]{ntheorem} % un théoreme

\usepackage{algorithm}
\usepackage{algorithmic}

\makeatletter
\renewcommand{\ALG@name}{Algorithme}
\makeatother

\usepackage[all]{xy}
%
\usepackage{listings}

\usepackage{hyperref}
\hypersetup{
    bookmarks=true,         % show bookmarks bar?
    unicode=false,          % non-Latin characters in Acrobat’s bookmarks
    pdftoolbar=true,        % show Acrobat’s toolbar?
    pdfmenubar=true,        % show Acrobat’s menu?
    pdffitwindow=false,     % window fit to page when opened
    pdfstartview={FitH},    % fits the width of the page to the window
    pdfnewwindow=true,      % links in new PDF window
    colorlinks=true,       % false: boxed links; true: colored links
    linkcolor=blue,          % color of internal links (change box color with linkbordercolor)
    citecolor=green,        % color of links to bibliography
    filecolor=magenta,      % color of file links
    urlcolor=cyan           % color of external links
}

%\let\urlorig\url
%\renewcommand{\url}[1]{%
%   \begin{otherlanguage}{english}\urlorig{#1}\end{otherlanguage}%
%}


% Bibliographie dans la Table des matières
\usepackage[nottoc, notlof, notlot]{tocbibind}



\usepackage{tikz}               % Tracer des graphes
\tikzset{every picture/.style={execute at begin picture={
\shorthandoff{:;!?};}
}}


\usepackage{thumbpdf}           % Fichier pdf généré 
                                % contien une              
                                % miniature de chaque slide 

 \usepackage{dsfont}
\usepackage{lastpage}           % Avoir total de pages dans le footer.
\usepackage{graphicx}           % Pour les images et figures
\usepackage{wrapfig}            % Pour détourer les figures
\usepackage{fancybox}           % De chouettes encadrements
\usepackage{lettrine}           % Pour de beaux paragraphes
\usepackage{setspace}           % Pour changer l'interligne
\usepackage{eurosym}            % Pour le signe \euro
\usepackage{xcolor}             % Pour mettre d la couleur
\usepackage{colortbl}           % Couleur dans tableaux
\usepackage{tabularx}           % pour des tableaux à taille de la page
\usepackage{longtable}          % Pour les grands tableaux
\usepackage[tight]{shorttoc}    % Pour faire un sommaire à la française.
\newcommand{\sommaire}{\shorttoc{Sommaire}{1}}
\usepackage{array}              % De beaux tableaux
\usepackage{multirow}           % Tableaux sur plusieurs lignes
                                % \multirow{nlignes}{largeur ou *}{contenu}
\usepackage{amsmath}            % Un peu de maths
\usepackage{amssymb}            % Encore des maths
\usepackage{mathtools}
\usepackage{empheq}             % Pour encadrer les équations


\usepackage{stmaryrd}			% pour les crochets d'intervalles entiers 


\usepackage{pgfpages}           % Pour avoir 2 pages sur A4 paysage
\usepackage{datetime}           % Jouer facilement avec les dates
% Pour la physique : 
\usepackage{numprint}           % Pour faire des groupes de 3 nombres
\usepackage[squaren,Gray,cdot]{SIunits}

\usepackage{pgf}
\usetikzlibrary{arrows}

\usepackage{nameref} %pour les entetes 
\makeatletter
\newcommand*{\currentname}{\@currentlabelname}
\makeatother

\definecolor{vert}{rgb}{0,0.6,0}
\definecolor{mauve}{rgb}{0.58,0,0.82}

\lstset{ %
  backgroundcolor=\color{white},   % choose the background color
  breaklines=true,                 % automatic line breaking only at whitespace
  captionpos=b,                    % sets the caption-position to bottom
  commentstyle=\color{vert},    % comment style
  escapeinside={\%*}{*)},          % if you want to add LaTeX within your code
  keywordstyle=\color{blue},       % keyword style
  stringstyle=\color{mauve},     % string literal style
  basicstyle=\small\ttfamily,%
  frame=single,
  %extendedchars=true,
 literate=%
         {é}{{\'e}}1
         {É}{{\'E}}1
         {à}{{\`a}}1
         {ê}{{\^e}}1
         {è}{{\`e}}1
}

%%%
% Commandes
\newcommand{\ndiv}{\nmid} %x\ndiv y <==> x ne divise pas y
\newcommand{\modulo}[3]{#1\equiv #2 \;[#3]}
\newcommand{\nmodulo}[3]{#1\not\equiv #2 \;[#3]}


% Ajout d'une image avec label
\newcommand{\image}[3]{
% \image{fichier}{label}{description}
\begin{center}
% Nécessite le package float
\begin{figure}[H]
\includegraphics[width=0.9\textwidth]{#1}
\caption{\label{#2}{#3}}
\end{figure}
\end{center}
}

% Ajout d'une image largeur page avec label
\newcommand{\imagebig}[3]{
% \imagebig{fichier}{label}{description}
\begin{center}
% Nécessite le package float
\begin{figure}[H]
\includegraphics[width=\textwidth]{#1}
\caption{\label{#2}{#3}}
\end{figure}
\end{center}
}

% Une image qui prend toute la page
\newcommand{\imagefull}[1]{
    \newgeometry{margin=0cm}
\begin{center}
\begin{figure}[H]
\includegraphics[width=0.96\paperwidth]{#1}
\end{figure}
\end{center}
\restoregeometry
\nopagebreak
}

% Un encadré grisé
\newcommand{\encadregris}[1]{
\begin{center}
\colorbox{gray!20}{
\begin{minipage}{0.95\textwidth}
{#1}
\end{minipage}
}
\end{center}
}

% Un mot grisé
\newcommand{\motgris}[1]{
\colorbox{gray!20}{{#1}}
}

% Un encadré
\newcommand{\encadre}[1]{
\begin{center}
\fbox{
\begin{minipage}{\textwidth}
{#1}
\end{minipage}
}
\end{center}
}

%Une boite coloré
\newenvironment{colbox}[1]
{\def\FrameCommand{\colorbox{#1}}%
   \MakeFramed{\advance\hsize-\width \FrameRestore}}
{\endMakeFramed}

\colorlet{shadecolor}{blue!8}

%%%%%%%%%% Des Maths %%%%%%%%%%%%%%

\newcommand{\RR}{\ensuremath{\mathbb{R}}}
\newcommand{\CC}{\ensuremath{\mathbb{C}}}
\newcommand{\ZZ}{\ensuremath{\mathbb{Z}}}
\newcommand{\QQ}{\ensuremath{\mathbb{Q}}}
\newcommand{\NN}{\ensuremath{\mathbb{N}}}
\newcommand{\PP}{\ensuremath{\mathbb{P}}}
\newcommand{\KK}{\ensuremath{\mathbb{K}}}
\newcommand{\EE}{\ensuremath{\mathbb{E}}}
\renewcommand{\SS}{\ensuremath{\mathbb{S}}}
\newcommand{\TT}{\ensuremath{\mathbb{T}}}

\newcounter{theo}[section] % créer un nouveau compteur
\renewcommand\thetheo{\thesection.\arabic{theo}}
\newcounter{prop}[section]
\renewcommand\theprop{\thesection.\arabic{prop}}

\newenvironment{theoreme}[1]
{\refstepcounter{theo}\begin{shaded}\textbf{Théorème \thetheo~ : } \textit{#1}\vspace{0.3em}
\hrule
\medbreak}
{\end{shaded}}


\newenvironment{propriete}
{\begin{shaded}\textbf{Propriété : }\\ }
{\end{shaded}}

\newenvironment{proposition}
{\refstepcounter{prop}\begin{shaded}\textbf{Proposition \theprop~ : }\\ }
{\end{shaded}}

\newenvironment{corollaire}
{\begin{shaded}\textbf{Corollaire : }\\ }
{\end{shaded}}


\newenvironment{lemme}
{\begin{shaded}\textbf{Lemme : }\\ }
{\end{shaded}}


\newenvironment{definition}[1]
{\stepcounter{theo}\begin{shaded}\textbf{Définition \thetheo~ : } \textit{#1}\vspace{0.3em}
\hrule
\medbreak}
{\end{shaded}}


\newenvironment{boxeq}{\setlength{\fboxsep}{15pt}
\setlength{\mylength}{\linewidth}%
\addtolength{\mylength}{-2\fboxsep}%
\addtolength{\mylength}{-2\fboxrule}%
\Sbox
\minipage{\mylength}%
\setlength{\abovedisplayskip}{0pt}%
\setlength{\belowdisplayskip}{0pt}%
\equation}%
{\endequation\endminipage\endSbox
\[\fbox{\TheSbox}\]}
  

\newcounter{exos}
\newcommand{\exo}[1]{\stepcounter{exos}\Ovalbox{Exercice \theexos} \textbf{#1}}
\newcommand{\attention}[1]{\textcolor{red!90}{\textbf{Attention :} #1 }}
\newcommand{\remarque}[1]{\textcolor{blue!90}{\textbf{Remarque :} #1 }}
\newcommand{\exemple}[1]{\textcolor{magenta!80}{\textbf{Exemple :} #1 }}
\newcommand{\remarques}[1]{\textcolor{blue!90}{\textbf{Remarques :} #1 }}


\newenvironment{preuve}
{\begin{leftbar}\textbf{Preuve :} \\ }
{\hfill\ensuremath{\Box}\end{leftbar}\medbreak}




% flèche
\newcommand{\ra}[0]{
    $\rightarrow$
}




% Image détourée
% \wrapimg{align}{width}{img}
\newcommand{\wrapimg}[3]{
\begin{wrapfigure}{#1}{#2}
\includegraphics[width={#2}]{#3}
\end{wrapfigure}
}

\newcommand{\hdr}[0]{
    \hdashrule{1cm}{1pt}{1pt}
}

\newcommand*{\etoile}
{
\begin{center}
*\par
*\hspace*{3ex}*
\end{center}
}


% listes avec puces carrées
\newcommand{\carlst}[1]{
    \begin{itemize}
    \renewcommand\labelitemi{\petitcarre}
    {#1}
    \end{itemize}
}


\DeclareMathOperator*{\argmin}{arg\,min}
\newcommand{\interior}[1]{%
  {\kern0pt#1}^{\mathrm{o}}%
}


\usepackage{tikz,tkz-tab}
\csname @addtoreset\endcsname{section}{part} 

\newcommand{\fonction}[5]{\ensuremath{\begin{array}[t]{l|ccl}
#1: & #2 & \longrightarrow & #3 \\
    & #4 & \longmapsto & #5 \end{array}}}






% MISE EN FORME DU TITRE
\makeatletter
\renewcommand{\maketitle}{
\begin{minipage}[l]{.8\linewidth}
\includegraphics[width=120px]{img/logo.png} %logo insa
\end{minipage} \hfill
\begin{minipage}[r]{.46\linewidth}
\includegraphics[width=100px]{img/logo_univ.png}
\end{minipage}

\vspace{1cm}
\begin{center}
\boxput*(0,1){\colorbox{white}{Projet de Fin d’Études}}{
\setlength{\fboxsep}{10pt}
\framebox[\textwidth]{
\begin{minipage}{8cm}
\vspace{0.2cm}
\center
\Large
\@title
\end{minipage}
}}
\end{center}

\vspace{1cm}
\begin{center}
Département de Génie Mathématique\\
Semestre 9 - \dateDoc
\end{center}

\vspace{2cm}
\includegraphics[width=\linewidth]{img/main.png}

\vfill
\begin{minipage}[l]{0.4\textwidth}
\large
\emph{Auteur :}\\
\@author\\
\end{minipage}
\hfill
\begin{minipage}[r]{0.4\textwidth}
\large
\flushright
\emph{A l'attention de :} \\
Carole \textsc{Le Guyader}\\ % Supervisor's Name
Vincent \textsc{Duval}\\
\end{minipage}
\newpage
}
\makeatother
%%%


%\begin{center}
%\vspace{2ex}
%{\huge \textsc{\@title}}
%\vspace{1ex}
%\\
%\linia\\
%\@author \hfill \@date
%\vspace{4ex}
%\end{center}

%\begin{tcolorbox}[enhanced,attach boxed title to top center={yshift=-3mm,yshifttext=-1mm},
 % colback=blue!5!white,colframe=blue!75!black,colbacktitle=red!80!black,
  %title=My title,fonttitle=\bfseries,
  %boxed title style={size=small,colframe=red!50!black} ]
  %This box uses a \textit{boxed title}. The box of the title can
  %be formatted independently from the main box.
%\end{tcolorbox}







